This section presents practical use case scenarios that demonstrate how Well MEing will serve its diverse user base in real-life situations. 
These scenarios are based on recurring themes and needs identified in our user research, including flexible habit customization, minimal input effort, AI-driven feedback, and social engagement. 
The use cases will highlight how different user, from fitness enthusiasts to casual wellness trackers, will interact with the app to set goals, log progress and receive insights. 
Each scenario will provide a narrative of user interaction with key features, guiding the design of intuitive workflows.

\section{Custom Habit Creation and Quick Logging}

\textbf{Related Features:} Custom habit creation, Quick habit logging, Minimalistic UI

\vspace{0.3cm}
\noindent
Anna wants to track her daily water intake and study hours. She opens \textit{Well MEing}, taps ``Create Habit,'' and defines two custom habits: ``Drink Water'' (measured in glasses per day) and ``Study Time'' (hours per day). She selects intuitive input types like sliders and number fields. Thanks to the clean and minimal UI, she completes setup in under a minute. Each day, she logs her progress with just two taps from the dashboard, keeping her tracking efficient and stress-free.

\vspace{0.2cm}
\noindent
\textbf{Goal:} Enable users to define and monitor personal wellness habits with minimal friction and high flexibility.

\section{AI-Generated Progress Reports and Personalized Feedback}

\textbf{Related Features:} AI-generated progress reports, Adaptable data visualization, Configurable notifications

\vspace{0.3cm}
\noindent
Luca uses \textit{Well MEing} to monitor his sleep and stress levels. Each week, the app generates a personalized AI-driven report summarizing his trends: ``Average sleep: 7 hours/night. Higher stress detected on days with less than 6 hours of sleep.'' The report includes graphs and bar charts, and provides insights such as: ``Try reducing screen time before bed.'' Luca customizes his notifications to receive a reminder at 10 PM for sleep preparation, avoiding unnecessary alerts.

\vspace{0.2cm}
\noindent
\textbf{Goal:} Offer users tailored insights that help them reflect on and improve their well-being, while maintaining control over notification frequency.

\section{Check the Progress of a Tracked Activity}

\textbf{Related Features:} Adaptable data visualization, Custom habit creation, Minimalistic UI

\vspace{0.3cm}
\noindent
Sofia has been tracking her daily step count and hydration over the past month. She opens \textit{Well MEing} and navigates to the statistics section, where she views a bar chart showing her daily step count over time. She notices a trend: her step count is higher on weekdays than weekends. The hydration graph shows that she consistently meets her target of 8 glasses/day. She uses this information to adjust her weekend routine to include more physical activity.

\vspace{0.2cm}
\noindent
\textbf{Goal:} Allow users to visualize and analyze their progress in a clear, personalized way, encouraging data-driven habit adjustments.
